\documentclass{article}

% Packages:
% For adjusting page geometry
\usepackage[
  % Set margins without considering header and footer
  ignoreheadfoot,
  % Seperation between body and page edge from the top
  top=2 cm,
  % Seperation between body and page edge from the bottom
  bottom=2 cm,
  % Seperation between body and page edge from the left
  left=2 cm,
  % Seperation between body and page edge from the right
  right=2 cm,
  % Seperation between body and footer
  footskip=1.0 cm,
  % Showframe % for debugging
]{geometry}
% For customizing section titles
\usepackage{titlesec}
% For making tables with fixed width columns
\usepackage{tabularx}
% Tabularx requires this
\usepackage{array}
% For coloring text
\usepackage[dvipsnames]{xcolor}
% Define primary color
\definecolor{primaryColor}{RGB}{0, 0, 0}
% For customizing lists
\usepackage{enumitem}
% For using icons
\usepackage{fontawesome5}
% For math
\usepackage{amsmath}
% For links, metadata and bookmarks
\usepackage[
    pdftitle={Saku Glumoff},
    pdfauthor={Saku Glumoff},
    pdfcreator={pdflatex},
    colorlinks=true,
    urlcolor=primaryColor
]{hyperref}
% For floating text on the page
\usepackage[pscoord]{eso-pic}
% For calculating lengths
\usepackage{calc}
% For bookmarks
\usepackage{bookmark}
% For getting the total number of pages
\usepackage{lastpage}
% For one column entries (adjustwidth environment)
\usepackage{changepage}
% For two and three column entries
\usepackage{paracol}
% For conditional statements
\usepackage{ifthen}
% For avoiding page brake right after the section title
\usepackage{needspace}

% Check if engine is pdflatex, xetex or luatex
\usepackage{iftex}
% Ensure that generate pdf is machine readable/ATS parsable:
\ifPDFTeX
    \input{glyphtounicode}
    \pdfgentounicode=1
    \usepackage[T1]{fontenc}
    \usepackage[utf8]{inputenc}
    \usepackage{lmodern}
\fi

\usepackage{charter}

% Some settings:
\raggedright
% Remove space before adjustwidth environment
\AtBeginEnvironment{adjustwidth}{\partopsep0pt}
% No header or footer
\pagestyle{empty}
% No section numbering
\setcounter{secnumdepth}{0}
% No indentation
\setlength{\parindent}{0pt}
% No top skip
\setlength{\topskip}{0pt}
% Set column seperation
\setlength{\columnsep}{0.15cm}
% No page numbering
\pagenumbering{gobble}

\titleformat{\section}{\needspace{4\baselineskip}\bfseries\large}{}{0pt}{}[\vspace{1pt}\titlerule]

% Section title spacing
\titlespacing{\section}{
    % Left space:
    -1pt
}{
    % Top space:
    0.3 cm
}{
    % Bottom space:
    0.2 cm
}

% Custom bullet points
\renewcommand\labelitemi{$\vcenter{\hbox{\small$\bullet$}}$}

% New environment for highlights
\newenvironment{highlights}{
    \begin{itemize}[
        topsep=0.10 cm,
        parsep=0.10 cm,
        partopsep=0pt,
        itemsep=0pt,
        leftmargin=0 cm + 10pt
    ]
}{
    \end{itemize}
}

% New environment for highlights for bullet entries
\newenvironment{highlightsforbulletentries}{
    \begin{itemize}[
        topsep=0.10 cm,
        parsep=0.10 cm,
        partopsep=0pt,
        itemsep=0pt,
        leftmargin=10pt
    ]
}{
    \end{itemize}
}

% New environment for one column entries
\newenvironment{onecolentry}{
    \begin{adjustwidth}{
        0 cm + 0.00001 cm
    }{
        0 cm + 0.00001 cm
    }
}{
    \end{adjustwidth}
}

% New environment for two column entries
\newenvironment{twocolentry}[2][]{
    \onecolentry
    \def\secondColumn{#2}
    \setcolumnwidth{\fill, 4.5 cm}
    \begin{paracol}{2}
}{
    \switchcolumn \raggedleft \secondColumn
    \end{paracol}
    \endonecolentry
}

% New environment for three column entries
\newenvironment{threecolentry}[3][]{
    \onecolentry
    \def\thirdColumn{#3}
    \setcolumnwidth{, \fill, 4.5 cm}
    \begin{paracol}{3}
    {\raggedright #2} \switchcolumn
}{
    \switchcolumn \raggedleft \thirdColumn
    \end{paracol}
    \endonecolentry
}

% New environment for the header
\newenvironment{header}{
    \setlength{\topsep}{0pt}\par\kern\topsep\centering\linespread{1.5}
}{
    \par\kern\topsep
}

% \placetextbox{<horizontal pos>}{<vertical pos>}{<stuff>}
\newcommand{\placelastupdatedtext}{
  % Add <stuff> to current page foreground
  \AddToShipoutPictureFG*{
    \put(
        \LenToUnit{\paperwidth-2 cm-0 cm+0.05cm},
        \LenToUnit{\paperheight-1.0 cm}
    ){\vtop{{\null}\makebox[0pt][c]{
        \small\color{gray}\textit{Last updated in September 2024}\hspace{\widthof{Last updated in September 2024}}
    }}}
  }
}

% Save the original href command in a new command
\let\hrefWithoutArrow\href

\begin{document}
  % New command for external links:
  \newcommand{\AND}{\unskip
      \cleaders\copy\ANDbox\hskip\wd\ANDbox
      \ignorespaces
  }
  \newsavebox\ANDbox
  \sbox\ANDbox{$|$}

  \begin{header}
      \fontsize{25 pt}{25 pt}
      \selectfont Saku Glumoff

      \vspace{5 pt}
      \normalsize
      \mbox{\faIcon{map-marker-alt} Oulu, Finland}
      \kern 5.0 pt
      \AND
      \kern 5.0 pt
      \mbox{\faIcon{envelope} \hrefWithoutArrow{mailto:saku.glumoff@gmail.com}{saku.glumoff@gmail.com}}
      \kern 5.0 pt
      \AND
      \kern 5.0 pt
      \mbox{\faIcon{phone-alt} \hrefWithoutArrow{tel:+358 50 369 3654}{+358 50 369 3654}}
      \kern 5.0 pt
      \AND
      \kern 5.0 pt
      \mbox{\faIcon{github} \hrefWithoutArrow{https://github.com/SakuGlumoff}{github.com/SakuGlumoff}}
      \vspace{5 pt}
  \end{header}
  \vspace{5 pt - 0.3 cm}
  \section{Education}
    \begin{twocolentry}{2014}
      \textbf{OSAO Kaukovainio liiketalous}, Datanomi
    \end{twocolentry}
    \vspace{0.1 cm}
    \begin{onecolentry}
      \begin{highlights}
        \item IT support, networks, cloud servers, virtualization, domain services
      \end{highlights}
    \end{onecolentry}
    \vspace{0.3 cm}
    \begin{twocolentry}{2017}
      \textbf{Oulu University of Applied Sciences}, B.Eng. Information Technology, software development
    \end{twocolentry}
    \vspace{0.1 cm}
    \begin{onecolentry}
      \begin{highlights}
        \item Software design, hardware design, business design
      \end{highlights}
    \end{onecolentry}
  \section{Experience}
    \begin{twocolentry}{2016 -- 2017}
      \textbf{Software Developer} at Monidor
    \end{twocolentry}
    \vspace{0.1 cm}
    \begin{onecolentry}
      \begin{highlights}
        \item Developed software and UI/UX for embedded devices running on STMicroelectornics' and Nordic Semiconductor's microcontrollers.
        \item Created test automation tools for embedded devices using Python.
        \item Developed software for Android phones to interface with embedded devices using Bluetooth Low Energy.
        \item Performed user testing with actual end users.
      \end{highlights}
    \end{onecolentry}
    \vspace{0.3 cm}
    \begin{twocolentry}{2017 -- 2020}
      \textbf{Software Developer} at Huld
    \end{twocolentry}
    \vspace{0.1 cm}
    \begin{onecolentry}
      \begin{highlights}
        \item Developed software for embedded devices running on different microcontrollers.
        \item Prototyped devices using different radios (Bluetooth Low Energy, GSM, NB-IoT, etc.).
        \item Developed software for very small scale and low energy devices and set up a continuous integration environment for the project.
        \item Managed networking for the Oulu office. Managed DevOps tools company-wide.
      \end{highlights}
    \end{onecolentry}
    \vspace{0.3 cm}
    \begin{twocolentry}{2020 -- 2025}
      \textbf{Senior Engineer} at Bittium Wireless
    \end{twocolentry}
    \vspace{0.1 cm}
    \begin{onecolentry}
      \begin{highlights}
        \item Developed software for Android phones.
        \item Developed software for Nordic Semiconductor's devices to create sniffers for Bluetooth Low Energy environments.
        \item Developed medical software for embedded devices running on Microchip's microcontrollers with an emphasis on USB functionality.
        \item Developed software for embedded devices using multiple radios. Helped facilitate compliance testing for radios. Helped implement low power modes for different devices.
        \item Designed a PCB used for automating user actions.
      \end{highlights}
    \end{onecolentry}
  \section{Skills}
    \begin{onecolentry}
      \textbf{Hardware:} Schematic and layout design with KiCad, troubleshooting/measurements using oscilloscopes, logic analyzers, USB analyzers, multimeters, using microcontrollers from Atmel/Microchip/STMicroelectornics/Nordic Semiconductor/Texas Instruments and others.
    \end{onecolentry}
    \vspace{0.3 cm}
    \begin{onecolentry}
      \textbf{Software:} C (C11), C++ (C++17), ARM assembly, Python, Bash, Git, Qt, UWP, FreeRTOS, MbedOS, Debugging with Segger J-Link and others, Docker, Jenkins
    \end{onecolentry}
    \vspace{0.3 cm}
    \begin{onecolentry}
      \textbf{Office:} Microsoft Office suite (Word, Excel, Powerpoint, etc.) and LibreOffice suite.
    \end{onecolentry}
  \newpage
  \section{Projects}
    \begin{twocolentry}{
      \href{https://github.com/SakuGlumoff/iida_devkit}{\faIcon{github} SakuGlumoff/iida\_devkit}
    }
      \textbf{Iida devkit}
    \end{twocolentry}
    \vspace{0.1 cm}
    \begin{onecolentry}
      \begin{highlights}
        \item A PCB design exercise using KiCad for hardware and Zephyr for software where the emphasis is on ethical and green design.
        \item The result is an IoT device containing different sensors, a battery management system and an LTE Cat-M1 radio.
      \end{highlights}
    \end{onecolentry}
    \vspace{0.3 cm}
    \begin{twocolentry}{
      \href{https://github.com/SakuGlumoff/dynamic-jenkins-node}{\faIcon{github} SakuGlumoff/dynamic-jenkins-node}
    }
      \textbf{Dynamic Jenkins Node}
    \end{twocolentry}
    \vspace{0.1 cm}
    \begin{onecolentry}
      \begin{highlights}
        \item A node template for the Jenkins automation tool, which supports dynamic IPs.
        \item The result is a node for the Jenkins tool, which can be behind a NAT gateway or can have a dynamically changing IP address without having to reconfigure the node each time the outwards visible IP address changes.
      \end{highlights}
    \end{onecolentry}
    \vspace{0.3 cm}
    \begin{twocolentry}{
      \href{https://sakuglumoff.github.io/2019/10/18/Continuous-Integration-with-IAR-EW/}{\faIcon{github} Continuous integration with IAR EW}
    }
      \textbf{Continuous Integration with IAR Embedded Workbench}
    \end{twocolentry}
    \vspace{0.1 cm}
    \begin{onecolentry}
      \begin{highlights}
        \item A publication on how to implement continuous integration using Python and the IAR Embedded Workbench on Windows machines.
      \end{highlights}
    \end{onecolentry}
    \vspace{0.3 cm}
  \section{Languages}
    \begin{onecolentry}
      \textbf{Finnish:} Native
    \end{onecolentry}
    \vspace{0.1 cm}
    \begin{onecolentry}
      \textbf{English:} Excellent
    \end{onecolentry}
    \vspace{0.1 cm}
    \begin{onecolentry}
      \textbf{Swedish:} Fine
    \end{onecolentry}
    \vspace{0.1 cm}
    \begin{onecolentry}
      \textbf{Russian:} Fine
    \end{onecolentry}
    \vspace{0.1 cm}
  \section{Hobbies}
    \begin{highlights}
      \item Programming/electronics
      \item Learning the Ukrainian language
      \item Watching movies
      \item Gaming
    \end{highlights}
\end{document}
